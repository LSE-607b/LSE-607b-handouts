\documentclass[]{interact}
\usepackage[scaled]{berasans}
\renewcommand*\familydefault{\sfdefault}  %% Only if the base font of the document is to be sans serif
\usepackage[T1]{fontenc}
\usepackage{xcolor}
\usepackage{epstopdf}% To incorporate .eps illustrations using PDFLaTeX, etc.
\usepackage[caption=false]{subfig}% Support for small, `sub' figures and tables
%\usepackage[nolists,tablesfirst]{endfloat}% To `separate' figures and tables from text if required

%\usepackage[doublespacing]{setspace}% To produce a `double spaced' document if required
%\setlength\parindent{24pt}% To increase paragraph indentation when line spacing is doubled
%\setlength\bibindent{2em}% To increase hanging indent in bibliography when line spacing is doubled

\usepackage{mathrsfs}

\usepackage[numbers,sort&compress]{natbib}% Citation support using natbib.sty
\bibpunct[, ]{[}{]}{,}{n}{,}{,}% Citation support using natbib.sty
\renewcommand\bibfont{\fontsize{10}{12}\selectfont}% Bibliography support using natbib.sty

\theoremstyle{plain}% Theorem-like structures provided by amsthm.sty
\newtheorem{theorem}{Theorem}[section]
\newtheorem{lemma}[theorem]{Lemma}
\newtheorem{corollary}[theorem]{Corollary}
\newtheorem{proposition}[theorem]{Proposition}

\theoremstyle{definition}
\newtheorem{definition}[theorem]{Definition}
\newtheorem{example}[theorem]{Example}
\newtheorem{assumption}[theorem]{Assumption}

\theoremstyle{remark}
\newtheorem{remark}{Remark}
\newtheorem{notation}{Notation}


\begin{document}
	
	
\title{Useful Inequalities in Probability and Statistics}

\author{
	\name{Pietro Maria Sparago\textsuperscript{a}\thanks{CONTACT P.~M. Author. Email: p.sparago@lse.ac.uk}}
	\affil{\textsuperscript{a}Department of Statistics, London School of Economics and Political Science, London, UK}
}

\maketitle

\section{Tails of the normal distribution}

\begin{theorem}  
\label{th1}	
Let $X \sim \mathcal{N}(0,1)$. Define $\overline{\Phi}(x):=P(X>x),x\geq 0$. Then
$$\boxed{\frac{1}{2}e^{-x\sqrt{2/\pi}-x^2/2}\leq \overline{\Phi}(x)\leq \min\bigg(\frac{1}{2}e^{-x^2/2},\frac{1}{x}\phi(x)\bigg),\,x\geq 0}$$
\end{theorem}
\begin{proof} We have, by continuity and Corollary \ref{corol_th1}:
$$2\overline{\Phi}(x)=\lim_{y\downarrow 0}\frac{\overline{\Phi}(x+y)}{\overline{\Phi}(y)}
\begin{cases}
\leq \lim_{y\downarrow 0}e^{-xy-x^2/2}=e^{-x^2/2}\\
\geq \lim_{y\downarrow 0}e^{-\rho(y)x-x^2/2}=e^{-x\sqrt{2/\pi}-x^2/2}
\end{cases}
$$
We conclude for the upper bound with Lemma \ref{lemma1_th1}.
\end{proof}

\begin{remark} Note that Proposition \ref{th1} yields a stronger version of the infimum Chernoff bound which is found with
$$\begin{aligned}\overline{\Phi}(x)\leq \inf_{\lambda \geq 0}e^{-\lambda x}e^{\lambda^2/2}=e^{-x^2/2},\,x\geq 0
\end{aligned}$$
where the second equality follows from the quadratic minimization condition $\frac{d}{d\lambda}(-\lambda x+\lambda^2/2)=0$ which yields $\lambda=x$.
\end{remark}

\subsection{Lemmas for the proof of Theorem \ref{th1} - mostly from \cite{PollardUGMTP}; Appendix D}

\begin{lemma} 
\label{lemma1_th1}	
Let $\phi(x):=(2\pi)^{-1/2}e^{-x^2/2}$. Then:
\begin{equation}
\bigg(\frac{1}{x}-\frac{1}{x^3}\bigg)\phi(x)<\overline{\Phi}(x)<\frac{1}{x}\phi(x),\,x>0
\end{equation}
\end{lemma}
\begin{proof} We have $\overline{\Phi}(x)=\int_x^\infty \phi(t)dt$. Note that for all $t>0$ the functions $1-3/t^4<1$ and $1+1/t^2>1$ so $(1-3/t^4)\phi(t)<\phi(t)<(1+1/t^2)\phi(t)$. We thus integrate from $x>0$ to infinity and obtain the claimed bounds.
\end{proof}
It follows immediately that for the function $\rho(x):=\phi(x)/\overline{\Phi}(x),\,x>0$ we have the bounds
$$x<\rho(x)<\frac{x^3}{x^2-1},\,x>1$$

\begin{theorem}
	\label{rhofunction}
	$\rho$ is increasing, $\rho(-\infty)=0$ and $\rho(0)=\sqrt{2/\pi}$. The function $\rho(x)-x$ decreases to zero as $x$ tends to infinity. The function $\ln \rho(x)$ is concave and $\ln \rho(x+\delta)<\ln \rho (x)+(\rho(x)-x)\delta$ for $x \in \mathbb{R}$ and $\delta >0$.
\end{theorem}
\begin{proof}
	We have
	$$\begin{aligned}\frac{1}{\rho(x)}=\frac{\overline{\Phi}(x)}{\phi(x)}&=\int_x^\infty\frac{\phi(t)}{\phi(x)}dt\\
	&\stackrel{z=t-x}=\int_0^\infty\frac{\phi(z+x)}{\phi(x)}dz\\
	&=\int_0^\infty e^{-zx}e^{-z^2/2}dz
	\end{aligned}$$
	That is, the Laplace transform of the finite measure $\mu(dz) := e^{-z^2/2}dz,z>0$. From this it follows that $\rho(0)=\sqrt{2/\pi}$ and that $\rho(-\infty)=0$, since by monotone convergence $\lim_{x\to -\infty}\frac{1}{\rho(x)}=\infty$. For $x\geq 0$ we have $ z e^{-zx}\leq z,z>0$ and $z \in L^1(d\mu)$. So by differentiation-under-integral lemma we have
	$$\begin{aligned}\rho(x)\bigg(\frac{d}{dx}\frac{1}{\rho(x)}\bigg)&=\rho(x)\bigg(-\int_0^\infty ze^{-zx}\mu(dz)\bigg)<0\\
	\rho(x)\bigg(\frac{d^2}{dx^2}\frac{1}{\rho(x)}\bigg)&=\rho(x)\int_0^\infty z^2 e^{-zx}\mu(dz)>0\\
	\end{aligned}$$
	The function $1/\rho$ is decreasing because $\frac{d}{dx}\frac{1}{\rho(x)}<0$, so that $\rho$ is increasing. Now, we have:
	$$\begin{aligned}\ln \rho(x)=\ln(\phi(x))-\ln(\overline{\Phi}(x))=-\ln(\sqrt{2\pi})-\frac{x^2}{2}-\ln(\overline{\Phi}(x))
	\end{aligned}$$
	Therefore 
	$$\frac{d}{dx}\ln \rho(x)=-x-\frac{1}{\overline{\Phi}(x)}\frac{d}{dx}\overline{\Phi}(x)=\frac{\phi(x)}{\overline{\Phi}(x)}-x=\rho(x)-x$$
	But we also have $\frac{d}{dx}(-\ln \rho(x))=\frac{d}{dx}\ln (1/\rho(x))$ so
	$$\begin{aligned}\frac{d}{dx}(\rho(x)-x)&=-\frac{d^2}{dx^2}\ln\frac{1}{\rho(x)}\\
	&=-\frac{d}{dx}\bigg(\rho(x)\bigg(\frac{d}{dx}\frac{1}{\rho(x)}\bigg)\bigg)\\
	&=-\bigg(\rho(x)\bigg(\frac{d^2}{dx^2}\frac{1}{\rho(x)}\bigg)+\bigg(\bigg(\frac{d}{dx}\rho(x)\bigg)\bigg(\frac{d}{dx}\frac{1}{\rho(x)}\bigg)\bigg)
	\end{aligned}$$
	and
	$$\frac{d}{dx}\frac{1}{\rho(x)}=-\frac{d}{dx}\rho(x)\frac{1}{\rho(x)^2}\implies \frac{d}{dx}\rho(x)=-\rho(x)^2\bigg(\frac{d}{dx}\frac{1}{\rho(x)}\bigg)$$
	Therefore 
	$$\frac{d}{dx}(\rho(x)-x)=-\bigg(\rho(x)\bigg(\frac{d^2}{dx^2}\frac{1}{\rho(x)}\bigg)-\rho(x)^2\bigg(\frac{d}{dx}\frac{1}{\rho(x)}\bigg)^2\bigg)$$
	The term in the parenthesis in the variance of the measure $\nu_x(dz):=\rho(x)e^{-zx}\mu(dz)$, which is strictly positive. In fact for any $x\geq 0$, $\nu_x(dz)$ is a probability measure:
	$$\int_0^\infty\nu_x(dz)=\rho(x)\int_0^\infty e^{-zx-z^2/2}dz=\rho(x)\int_x^\infty\frac{\phi(t)}{\phi(x)}dt=1$$
	Thus, $\rho(x)-x$ is decreasing because its first derivative is strictly negative. Now since
	$$\begin{aligned}
	\frac{d^2}{dx^2}\ln \rho(x)&=\frac{d}{dx}(\rho(x)-x)<0\\
	\end{aligned}$$
	then the function $\ln \rho(x)$ is thus concave. To see that $\rho(x)-x$ vanishes as $x\to \infty$:
	$$0<\rho(x)-x<\frac{x}{x^2-1}\stackrel{x\to \infty}{\to} 0$$
	It remains to show the last claim. For $\delta >0$ and some $x^*\in (x,x+\delta)$, since $\rho(x)-x$ is decreasing:
	$$\ln\frac{\rho(x+\delta)}{\rho(x)}=\delta \frac{\ln \rho(x+\delta)-\ln \rho(x)}{\delta}\stackrel{\textrm{MVT}}{=}\delta(\rho(x^*)-x^*)<\delta(\rho(x)-x)$$
	and we conclude.
\end{proof}

\begin{corollary}
\label{corol_th1}
For $x\geq 0$ and $\delta>0$:
$$e^{-\rho(x)\delta -\delta^2/2}\leq \frac{\overline{\Phi}(x+\delta)}{\overline{\Phi}(x)}\leq e^{-x\delta -\delta^2/2}$$
\end{corollary}
\begin{proof} We have
$$\begin{aligned}\overline{\Phi}(x+\delta)&=\int_{x}^\infty\phi(z+\delta)dz\\
&=\frac{e^{-\delta^2/2}}{\sqrt{2\pi}}\int_{x}^\infty e^{-z^2/2-z\delta}dz\\
&\stackrel{e^{-z\delta}\leq e^{-x\delta},\forall z\geq x}{\leq }\frac{e^{-\delta^2/2-x\delta}}{\sqrt{2\pi}}\int_{x}^\infty e^{-z^2/2}dz\\
&=e^{-\delta^2/2-x\delta}\overline{\Phi}(x)
\end{aligned}$$
By Theorem \ref{rhofunction} we have $
\ln\frac{\rho(x+\delta)}{\rho(x)}< (\rho(x)-x)\delta$. We also have
$$\begin{aligned}\frac{\overline{\Phi}(x+\delta)}{\overline{\Phi}(x)}&=\frac{\overline{\Phi}(x+\delta)}{\phi(x+\delta)}\frac{\phi(x)}{\overline{\Phi}(x)}\frac{\phi(x+\delta)}{\phi(x)}\\
&=\frac{\rho(x+\delta)}{\rho(x)}\frac{\phi(x+\delta)}{\phi(x)}\\
&=e^{-\ln\frac{\rho(x+\delta)}{\rho(x)}}e^{-\delta^2/2-x\delta}\\
&\geq e^{-\rho(x)\delta -\delta^2/2}
\end{aligned}$$
and we conclude.
\end{proof}


\subsection{Law of iterated logarithm - mostly from \cite{PollardUGMTP}, Chapters 11.1 and 11.2}

\begin{lemma} 
\label{gaussian_maximal_inequality}	
Let $S_n=X_1+...+X_n$ for $(X_n)_{n\leq N}$ IID $\sim \mathcal{N}(0,1)$. Then for $x\geq 0$
$${P(\max_{n\leq N}S_n\geq x)\leq  2P(S_N\geq x)}$$
\end{lemma}
\begin{proof} We have $P(S_N-S_n\geq 0)=1/2$ for any $n<N$ by the fact that $S_N-S_n\sim \mathcal{N}(0,N-n)$. Define $\tau=\inf\{n\leq N:S_n\geq x\}\wedge N$. The events $\{\tau=n\},n \in \{1,...,N\}$ are disjoint. So we have
$$\begin{aligned}P(\max_{n\leq N}S_n\geq x)&=P(\cup_{n\leq N}\{\tau=n\})\\
&= P(\cup_{n\leq N}(\{\tau=n\}\cap\{S_n\geq x\}))\\
&=\sum_{n\leq N}P(\{\tau=n\}\cap\{S_n\geq x\})\\
&=2\sum_{n< N}P(\{\tau=n\}\cap\{S_n\geq x\})P(S_N-S_n\geq 0)\\
&+P(\{\tau=N\}\cap\{S_N\geq x\})\\
&\stackrel{(*)}{=}2\sum_{n< N}P(\{\tau=n\}\cap\{S_n\geq x\}\cap\{S_N-S_n\geq 0\})\\
&+P(\{\tau=N\}\cap\{S_N\geq x\})\\
&\stackrel{(**)}{\leq} 2\sum_{n\leq N}P(\{\tau=n\}\cap\{S_N\geq x\})\\
&=2P(S_N\geq x)
\end{aligned}$$
where equality $(*)$ follows from the fact that $S_N-S_n$ is independent of $S_n$, while the inequality $(**)$ follows from the fact that $\{S_n\geq x\}\cap\{S_N-S_n\geq 0\}\subseteq \{S_N\geq x\}$ and $P(\{\tau=N\}\cap\{S_N\geq x\})\leq 2P(\{\tau=N\}\cap\{S_N\geq x\})$.
\end{proof}

\begin{theorem}
	\label{lil_gauss}
	Let $S_n=X_1+...+X_n$ where $(X_n)_{n \in \mathbb{N}}$ are IID $\sim \mathcal{N}(0,1)$. Then:
	\begin{enumerate}
		\item $\limsup_{n\to \infty}\frac{S_n}{\sqrt{2n \ln \ln n}}=1$ a.s.;
		\item $\liminf_{n\to \infty}\frac{S_n}{\sqrt{2n \ln \ln n}}=-1$ a.s.;
		\item $\frac{S_n}{\sqrt{2n \ln \ln n}}\in B$ i.o. a.s. for any open $B\subseteq [-1,1]$.
	\end{enumerate}	
\end{theorem}
\begin{proof} (i). Let $B$ be a block of consecutive positive integers (i.e. $B=\{n+1,...,n+k\}$ for some $n,k$). Let $\gamma >1$. Then by Lemma \ref{gaussian_maximal_inequality} and Theorem \ref{lemma1_th1}
$$\begin{aligned}
P(\cup_{\ell \in B}\{S_\ell\geq \gamma \sqrt{2\ell \ln \ln \ell}\})&\leq P(\max_{\ell \in B}S_\ell\geq \gamma \sqrt{2n \ln \ln n})\\
&\leq 2P(S_{n+k}\geq \gamma \sqrt{2n \ln \ln n})\\
&=2P\bigg(\frac{S_{n+k}}{\sqrt{n+k}}\geq \gamma\frac{\sqrt{2n \ln \ln n}}{\sqrt{n+k}}\bigg)\\
&=2\overline{\Phi}\bigg(\gamma\frac{\sqrt{2n \ln \ln n}}{\sqrt{n+k}}\bigg)\\
&\leq \exp\bigg(-\frac{1}{2}\gamma^2\frac{{2n \ln \ln n}}{n+k}\bigg)
\end{aligned}$$
Now consider a sequence of consecutive blocks $B_k=\{n:n_k<n\leq n_{k+1}\}$ so that $n_k/\rho^k\to 1$ for some constant $\rho\in (1,\gamma)$. By the above then we then have the bounds
$$\begin{aligned}P(\cup_{\ell \in B_k}\{S_\ell\geq \gamma \sqrt{2\ell \ln \ln \ell}\})&\leq \exp\bigg(-\frac{1}{2}\rho^2\frac{{2n_k \ln \ln n_k}}{n_{k+1}}\bigg)\\
&\color{red}{\stackrel{k\textrm{ large enough}}{\approx}}\color{black}{
\exp\bigg(-\frac{1}{2}\rho^2\frac{{2\rho^k \ln \ln \rho^k}}{\rho^{k+1}}\bigg)}\\
&=\exp\bigg(-\rho (\ln k + \ln \ln \rho)\bigg)\\
&=C(\rho)k^{-\rho}
\end{aligned}$$
which implies, by Borel-Cantelli I, that $P(S_n\geq \gamma \sqrt{2n \ln \ln n} \textrm{ i.o.})=0$. Therefore $P(\limsup_{n \to \infty}\frac{S_n}{\sqrt{2n \ln \ln n}}\leq \gamma )=1$ for any $\gamma>1$. We conclude that, considering a sequence $\gamma_\ell\downarrow 1$,
$$P\bigg(\limsup_{n \to \infty}\frac{S_n}{\sqrt{2n \ln \ln n}}\leq 1 \bigg)=1$$
We now need to prove that $\limsup_{n \to \infty}\frac{S_n}{\sqrt{2n \ln \ln n}}\geq 1$
	
\end{proof}

\newpage 

\begin{thebibliography}{99}
	
	\bibitem[Pollard, 2001]{PollardUGMTP}
    Pollard, D. (2001).
	\newblock {\em A {U}ser's {G}uide to {M}easure {T}heoretic {P}robability}.
	\newblock Cambridge University Press.
	
\end{thebibliography}


\end{document}